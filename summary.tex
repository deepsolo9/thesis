\chapter*{Summary}
\addcontentsline{toc}{chapter}{Summary}


In recent years, non-volatile memory like PCM has
been considered an attractive
alternative to flash memory and DRAM. 
It has promising features, including non-volatile storage,
byte addressability, fast read and write operations,
and supports random accesses. Many research scholars are working on designing adaptive systems based on such memory technologies. 
However, there are also some challenges
in designing algorithms for this kind of non-volatile-based memory systems,
such as longer write latency and
higher energy consumption compared to DRAM.
In this thesis, we will talk about our redesign of the indexing technique for traditional database systems for them.

We propose a new {\em predictive} \bplustree index, called the \bptree.
which is tailored for database systems that make use of PCM.
We know that the relative slow-write is one of the major challenges when designing algorithms for the new systems and thus 
our trivial target is to avoid the writes as many as possible during the execution of our algorithms. 
For \bplustree, we find that each time a node is full, we need to split the node and 
write half of the keys on this node to a new place which is the major source of the more writes during the construction of the tree. 
Our \bptree can reduce the data movements caused by tree node
splits and merges that arise from insertions and deletions.
This is achieved by pre-allocating space on the memory for near future data.
To ensure the space are allocated where they are needed,
we propose a novel predictive model
to ascertain future data distribution based on the current data.
In addition, as in \cite{chen2011rethinking},
when keys are inserted into a leaf node,
they are packed but need not be in sorted order which can also reduce some writes.

We implemented the \bptree
in PostgreSQL and evaluated it in an emulated environment.
Since we do not have any PCM product now, we need to simulate the environment in our experiments. 
We customized the buffer management and calculate the number of writes based on the cache line size. 
Besides the normal insertion, deletion and search performance, we also did experiments to see how sensitive
our \bptree is to the changes of the data distribution. 
Our experimental results show that the \bptree significantly reduces
the number of writes,
therefore making it write and energy efficient and suitable for a PCM-like
hardware environment. For the future work, besides the indexing technique, we can move on to make the query
processing more friendly to the next generation non-volatile memory.
